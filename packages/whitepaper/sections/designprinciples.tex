\section{Design Principles}

Taiko's ZK-Rollup design follows a few principles:

\begin{enumerate}
\item \textbf{Secure.} The design should prioritize security above all else.
\item \textbf{Minimal.} The design should be simple and focus only on the core ZK-Rollup protocol, not its upgradeability, governance, low-level optimizations, non-core bridging functionality, etc.
\item  \textbf{Robust.} The design should not depend on game theory for security. All security assumptions should be directly or indirectly enforced by Ethereum and the protocol. For example, there should be no need to use a Proof-of-Stake-like system to slash participants for bad behavior.
\item \textbf{Decentralized.} The design should encourage a high degree of decentralization in terms of block proposing and proving. No single party should be able to control all transaction ordering or be solely responsible for proving blocks. Being sufficiently decentralized implies that the protocol should keep working in a reliable manner in adversarial situations.
\item \textbf{Permissionless.} Anyone willing should be able to join and leave the network at any time, without causing significant disturbance to the network or being detrimental to the party in question. No single entity should have the power to allowlist or blocklist participants.
\item \textbf{Ethereum-Aligned.} The goal is to help Ethereum scale in the best possible way. Ether is used to pay the L2 transaction fees.
\item \textbf{Ethereum-Equivalent.} The design should stick to the design of Ethereum as closely as possible, not only for compatibility reasons but also for the expectations and demands of users of Ethereum L2 solutions.
\end{enumerate}

With these principles, our objective is to design and implement an Ethereum-equivalent (type-1) ZK-Rollup \cite{vitalik-zkevm}. This not only means that Taiko can directly interpret EVM bytecode, but also uses the same hash functions, state trees, transaction trees, precompiled contracts, and other in-consensus logic. We do however disable certain EIPs in the initial implementation\cite{taikoprotogithub} that will be re-enabled later (see Section \ref{sec:eips}).
