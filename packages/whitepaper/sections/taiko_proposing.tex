\subsection{Proposing Blocks}
Any willing entity can propose new Taiko blocks using the {\underline{TaikoL1}} contract. Blocks are appended to a list in the order they come in (which is dictated by Ethereum). Once the block is in the list it is verified and nodes can apply its state to the latest L2 state (see Section \ref{sec:properties}). Certain blocks however are deemed invalid by the protocol and these blocks will be ignored (see Section \ref{sec:proving-invalid}).

\subsubsection{Proposed Block} A proposed block in Taiko is the collection of information (known as the block's \emph{Metadata}), $C$, and a list of transactions, $L$,  (known as the block \emph{txList}). Formally, we can refer to a proposed block as $\dot{B}$:

\begin{eqnarray}
\dot{B} \equiv (\dot{B}_{\mathrm{C}}, \dot{B}_{\mathbf{L}}) \equiv (C, L)
\end{eqnarray}

\subsubsection{Block Metadata}\label{metadata} The block metadata, $C$, is a tuple of 9 items comprising:

\begin{description}
\item[id] A value equal to the number of proposed blocks. The genesis block has an id of zero; formally $C_{\mathrm{i}}$.
\item[beneficiary] The 20-byte address to which all transaction fees in the block will be transferred; formally $C_{\mathrm{c}}$.
\item[gasLimit] The total gas limit used by the block; formally $C_{\mathrm{l}}$.
\item[timestamp] The timestamp used in the block, set to the enclosing L1 timestamp; formally $C_{\mathrm{s}}$.
\item[mixHash] The mixHash value used in the block, set to the enclosing L1 mixHash; formally $C_{\mathrm{m}}$.
\item[extraData] The extraData value for the L2 block. This must be 32 bytes or fewer; formally $C_{\mathrm{x}}$.
\item[txListHash] The Keccak-256 hash of this block's txList; formally $C_{\mathrm{t}}$. 
\item[l1Height] The enclosing L1 block's parent block number; formally $C_{\mathrm{a}}$.
\item[l1Hash] The enclosing L1 block's parent block hash; formally $C_{\mathrm{h}}$.
\end{description}

\subsubsection{txList}\label{sec:txlist}
The txList is the RLP-serialised list of all the transactions in an L2 block. As future improvements like data sharding (see Section \ref{sec:datablobs}) and compression (see Section \ref{sec:compression}) will make this data less accessible from L1 smart contracts, we make sure not to depend on the actual data itself (except currently to calculate its hash). This will allow us to easily switch to other, more efficient, methods of storing this data on Ethereum. It is likely that it will be difficult to even bring this data back to an L1 smart contract because this is severely limited by the transaction data gas cost and the Ethereum block gas limit.

\subsubsection{Proposed Block Intrinsic Validity}
The proposed block must pass an \emph{Intrinsic Validity} test before it is accepted by the \underline{TaikoL1} contract. 

We are able to define the Intrinsic Validity function as:

\begin{eqnarray}
V^{b}(\dot{B}) & \equiv &   V^{b}(C,L)  \\
\nonumber & \equiv &   R_\mathbf{i} \le R_\mathbf{f} + K_\mathrm{MaxNumBlocks}   \quad \wedge \\
\nonumber& & \lVert L \rVert > 0 \quad \wedge \\
\nonumber & & \lVert L \rVert \le K_{\mathrm{TxListMaxBytes}} \quad \wedge \\
\nonumber& & C_{\mathbf{c}} \ne 0   \quad \wedge \\
\nonumber& & C_{\mathbf{i}} = R_\mathbf{i}   \quad \wedge \\
\nonumber& & C_{\mathbf{s}} = \texttt{TIMESTAMP}   \quad \wedge \\
\nonumber& & C_{\mathbf{m}} = \texttt{DIFFICULTY}   \quad \wedge \\
\nonumber& & C_{\mathbf{t}} \ne 0   \quad \wedge \\
\nonumber& & C_{\mathbf{t}} = \texttt{KEC}(L)   \quad \wedge \\
\nonumber& & C_{\mathbf{a}} = \texttt{NUMBER} - 1   \quad \wedge \\
\nonumber& & C_{\mathbf{h}} = \texttt{BLOCKHASH}(C_{\mathbf{a}})
\end{eqnarray}

After passing the test, the proposed block is appended to the proposed block list $R_\mathrm{b}$ and $R_\mathrm{i}$ is incremented by one.
